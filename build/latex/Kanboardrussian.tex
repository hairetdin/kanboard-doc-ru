% Generated by Sphinx.
\def\sphinxdocclass{report}
\documentclass[letterpaper,10pt,russian]{sphinxmanual}
\usepackage[utf8]{inputenc}
\DeclareUnicodeCharacter{00A0}{\nobreakspace}
\usepackage{cmap}
\usepackage[T1]{fontenc}
\usepackage{babel}

\usepackage[Sonny]{fncychap}
\usepackage{longtable}
\usepackage{sphinx}
\usepackage{multirow}


\addto\captionsrussian{\renewcommand{\figurename}{Fig. }}
\addto\captionsrussian{\renewcommand{\tablename}{Table }}
\floatname{literal-block}{Listing }



\title{Kanboard russian Documentation}
\date{01 July 2016}
\release{0.0.1}
\author{hairetdin}
\newcommand{\sphinxlogo}{}
\renewcommand{\releasename}{Выпуск}
\makeindex

\makeatletter
\def\PYG@reset{\let\PYG@it=\relax \let\PYG@bf=\relax%
    \let\PYG@ul=\relax \let\PYG@tc=\relax%
    \let\PYG@bc=\relax \let\PYG@ff=\relax}
\def\PYG@tok#1{\csname PYG@tok@#1\endcsname}
\def\PYG@toks#1+{\ifx\relax#1\empty\else%
    \PYG@tok{#1}\expandafter\PYG@toks\fi}
\def\PYG@do#1{\PYG@bc{\PYG@tc{\PYG@ul{%
    \PYG@it{\PYG@bf{\PYG@ff{#1}}}}}}}
\def\PYG#1#2{\PYG@reset\PYG@toks#1+\relax+\PYG@do{#2}}

\expandafter\def\csname PYG@tok@gd\endcsname{\def\PYG@tc##1{\textcolor[rgb]{0.63,0.00,0.00}{##1}}}
\expandafter\def\csname PYG@tok@gu\endcsname{\let\PYG@bf=\textbf\def\PYG@tc##1{\textcolor[rgb]{0.50,0.00,0.50}{##1}}}
\expandafter\def\csname PYG@tok@gt\endcsname{\def\PYG@tc##1{\textcolor[rgb]{0.00,0.27,0.87}{##1}}}
\expandafter\def\csname PYG@tok@gs\endcsname{\let\PYG@bf=\textbf}
\expandafter\def\csname PYG@tok@gr\endcsname{\def\PYG@tc##1{\textcolor[rgb]{1.00,0.00,0.00}{##1}}}
\expandafter\def\csname PYG@tok@cm\endcsname{\let\PYG@it=\textit\def\PYG@tc##1{\textcolor[rgb]{0.25,0.50,0.56}{##1}}}
\expandafter\def\csname PYG@tok@vg\endcsname{\def\PYG@tc##1{\textcolor[rgb]{0.73,0.38,0.84}{##1}}}
\expandafter\def\csname PYG@tok@vi\endcsname{\def\PYG@tc##1{\textcolor[rgb]{0.73,0.38,0.84}{##1}}}
\expandafter\def\csname PYG@tok@mh\endcsname{\def\PYG@tc##1{\textcolor[rgb]{0.13,0.50,0.31}{##1}}}
\expandafter\def\csname PYG@tok@cs\endcsname{\def\PYG@tc##1{\textcolor[rgb]{0.25,0.50,0.56}{##1}}\def\PYG@bc##1{\setlength{\fboxsep}{0pt}\colorbox[rgb]{1.00,0.94,0.94}{\strut ##1}}}
\expandafter\def\csname PYG@tok@ge\endcsname{\let\PYG@it=\textit}
\expandafter\def\csname PYG@tok@vc\endcsname{\def\PYG@tc##1{\textcolor[rgb]{0.73,0.38,0.84}{##1}}}
\expandafter\def\csname PYG@tok@il\endcsname{\def\PYG@tc##1{\textcolor[rgb]{0.13,0.50,0.31}{##1}}}
\expandafter\def\csname PYG@tok@go\endcsname{\def\PYG@tc##1{\textcolor[rgb]{0.20,0.20,0.20}{##1}}}
\expandafter\def\csname PYG@tok@cp\endcsname{\def\PYG@tc##1{\textcolor[rgb]{0.00,0.44,0.13}{##1}}}
\expandafter\def\csname PYG@tok@gi\endcsname{\def\PYG@tc##1{\textcolor[rgb]{0.00,0.63,0.00}{##1}}}
\expandafter\def\csname PYG@tok@gh\endcsname{\let\PYG@bf=\textbf\def\PYG@tc##1{\textcolor[rgb]{0.00,0.00,0.50}{##1}}}
\expandafter\def\csname PYG@tok@ni\endcsname{\let\PYG@bf=\textbf\def\PYG@tc##1{\textcolor[rgb]{0.84,0.33,0.22}{##1}}}
\expandafter\def\csname PYG@tok@nl\endcsname{\let\PYG@bf=\textbf\def\PYG@tc##1{\textcolor[rgb]{0.00,0.13,0.44}{##1}}}
\expandafter\def\csname PYG@tok@nn\endcsname{\let\PYG@bf=\textbf\def\PYG@tc##1{\textcolor[rgb]{0.05,0.52,0.71}{##1}}}
\expandafter\def\csname PYG@tok@no\endcsname{\def\PYG@tc##1{\textcolor[rgb]{0.38,0.68,0.84}{##1}}}
\expandafter\def\csname PYG@tok@na\endcsname{\def\PYG@tc##1{\textcolor[rgb]{0.25,0.44,0.63}{##1}}}
\expandafter\def\csname PYG@tok@nb\endcsname{\def\PYG@tc##1{\textcolor[rgb]{0.00,0.44,0.13}{##1}}}
\expandafter\def\csname PYG@tok@nc\endcsname{\let\PYG@bf=\textbf\def\PYG@tc##1{\textcolor[rgb]{0.05,0.52,0.71}{##1}}}
\expandafter\def\csname PYG@tok@nd\endcsname{\let\PYG@bf=\textbf\def\PYG@tc##1{\textcolor[rgb]{0.33,0.33,0.33}{##1}}}
\expandafter\def\csname PYG@tok@ne\endcsname{\def\PYG@tc##1{\textcolor[rgb]{0.00,0.44,0.13}{##1}}}
\expandafter\def\csname PYG@tok@nf\endcsname{\def\PYG@tc##1{\textcolor[rgb]{0.02,0.16,0.49}{##1}}}
\expandafter\def\csname PYG@tok@si\endcsname{\let\PYG@it=\textit\def\PYG@tc##1{\textcolor[rgb]{0.44,0.63,0.82}{##1}}}
\expandafter\def\csname PYG@tok@s2\endcsname{\def\PYG@tc##1{\textcolor[rgb]{0.25,0.44,0.63}{##1}}}
\expandafter\def\csname PYG@tok@nt\endcsname{\let\PYG@bf=\textbf\def\PYG@tc##1{\textcolor[rgb]{0.02,0.16,0.45}{##1}}}
\expandafter\def\csname PYG@tok@nv\endcsname{\def\PYG@tc##1{\textcolor[rgb]{0.73,0.38,0.84}{##1}}}
\expandafter\def\csname PYG@tok@s1\endcsname{\def\PYG@tc##1{\textcolor[rgb]{0.25,0.44,0.63}{##1}}}
\expandafter\def\csname PYG@tok@ch\endcsname{\let\PYG@it=\textit\def\PYG@tc##1{\textcolor[rgb]{0.25,0.50,0.56}{##1}}}
\expandafter\def\csname PYG@tok@m\endcsname{\def\PYG@tc##1{\textcolor[rgb]{0.13,0.50,0.31}{##1}}}
\expandafter\def\csname PYG@tok@gp\endcsname{\let\PYG@bf=\textbf\def\PYG@tc##1{\textcolor[rgb]{0.78,0.36,0.04}{##1}}}
\expandafter\def\csname PYG@tok@sh\endcsname{\def\PYG@tc##1{\textcolor[rgb]{0.25,0.44,0.63}{##1}}}
\expandafter\def\csname PYG@tok@ow\endcsname{\let\PYG@bf=\textbf\def\PYG@tc##1{\textcolor[rgb]{0.00,0.44,0.13}{##1}}}
\expandafter\def\csname PYG@tok@sx\endcsname{\def\PYG@tc##1{\textcolor[rgb]{0.78,0.36,0.04}{##1}}}
\expandafter\def\csname PYG@tok@bp\endcsname{\def\PYG@tc##1{\textcolor[rgb]{0.00,0.44,0.13}{##1}}}
\expandafter\def\csname PYG@tok@c1\endcsname{\let\PYG@it=\textit\def\PYG@tc##1{\textcolor[rgb]{0.25,0.50,0.56}{##1}}}
\expandafter\def\csname PYG@tok@o\endcsname{\def\PYG@tc##1{\textcolor[rgb]{0.40,0.40,0.40}{##1}}}
\expandafter\def\csname PYG@tok@kc\endcsname{\let\PYG@bf=\textbf\def\PYG@tc##1{\textcolor[rgb]{0.00,0.44,0.13}{##1}}}
\expandafter\def\csname PYG@tok@c\endcsname{\let\PYG@it=\textit\def\PYG@tc##1{\textcolor[rgb]{0.25,0.50,0.56}{##1}}}
\expandafter\def\csname PYG@tok@mf\endcsname{\def\PYG@tc##1{\textcolor[rgb]{0.13,0.50,0.31}{##1}}}
\expandafter\def\csname PYG@tok@err\endcsname{\def\PYG@bc##1{\setlength{\fboxsep}{0pt}\fcolorbox[rgb]{1.00,0.00,0.00}{1,1,1}{\strut ##1}}}
\expandafter\def\csname PYG@tok@mb\endcsname{\def\PYG@tc##1{\textcolor[rgb]{0.13,0.50,0.31}{##1}}}
\expandafter\def\csname PYG@tok@ss\endcsname{\def\PYG@tc##1{\textcolor[rgb]{0.32,0.47,0.09}{##1}}}
\expandafter\def\csname PYG@tok@sr\endcsname{\def\PYG@tc##1{\textcolor[rgb]{0.14,0.33,0.53}{##1}}}
\expandafter\def\csname PYG@tok@mo\endcsname{\def\PYG@tc##1{\textcolor[rgb]{0.13,0.50,0.31}{##1}}}
\expandafter\def\csname PYG@tok@kd\endcsname{\let\PYG@bf=\textbf\def\PYG@tc##1{\textcolor[rgb]{0.00,0.44,0.13}{##1}}}
\expandafter\def\csname PYG@tok@mi\endcsname{\def\PYG@tc##1{\textcolor[rgb]{0.13,0.50,0.31}{##1}}}
\expandafter\def\csname PYG@tok@kn\endcsname{\let\PYG@bf=\textbf\def\PYG@tc##1{\textcolor[rgb]{0.00,0.44,0.13}{##1}}}
\expandafter\def\csname PYG@tok@cpf\endcsname{\let\PYG@it=\textit\def\PYG@tc##1{\textcolor[rgb]{0.25,0.50,0.56}{##1}}}
\expandafter\def\csname PYG@tok@kr\endcsname{\let\PYG@bf=\textbf\def\PYG@tc##1{\textcolor[rgb]{0.00,0.44,0.13}{##1}}}
\expandafter\def\csname PYG@tok@s\endcsname{\def\PYG@tc##1{\textcolor[rgb]{0.25,0.44,0.63}{##1}}}
\expandafter\def\csname PYG@tok@kp\endcsname{\def\PYG@tc##1{\textcolor[rgb]{0.00,0.44,0.13}{##1}}}
\expandafter\def\csname PYG@tok@w\endcsname{\def\PYG@tc##1{\textcolor[rgb]{0.73,0.73,0.73}{##1}}}
\expandafter\def\csname PYG@tok@kt\endcsname{\def\PYG@tc##1{\textcolor[rgb]{0.56,0.13,0.00}{##1}}}
\expandafter\def\csname PYG@tok@sc\endcsname{\def\PYG@tc##1{\textcolor[rgb]{0.25,0.44,0.63}{##1}}}
\expandafter\def\csname PYG@tok@sb\endcsname{\def\PYG@tc##1{\textcolor[rgb]{0.25,0.44,0.63}{##1}}}
\expandafter\def\csname PYG@tok@k\endcsname{\let\PYG@bf=\textbf\def\PYG@tc##1{\textcolor[rgb]{0.00,0.44,0.13}{##1}}}
\expandafter\def\csname PYG@tok@se\endcsname{\let\PYG@bf=\textbf\def\PYG@tc##1{\textcolor[rgb]{0.25,0.44,0.63}{##1}}}
\expandafter\def\csname PYG@tok@sd\endcsname{\let\PYG@it=\textit\def\PYG@tc##1{\textcolor[rgb]{0.25,0.44,0.63}{##1}}}

\def\PYGZbs{\char`\\}
\def\PYGZus{\char`\_}
\def\PYGZob{\char`\{}
\def\PYGZcb{\char`\}}
\def\PYGZca{\char`\^}
\def\PYGZam{\char`\&}
\def\PYGZlt{\char`\<}
\def\PYGZgt{\char`\>}
\def\PYGZsh{\char`\#}
\def\PYGZpc{\char`\%}
\def\PYGZdl{\char`\$}
\def\PYGZhy{\char`\-}
\def\PYGZsq{\char`\'}
\def\PYGZdq{\char`\"}
\def\PYGZti{\char`\~}
% for compatibility with earlier versions
\def\PYGZat{@}
\def\PYGZlb{[}
\def\PYGZrb{]}
\makeatother

\renewcommand\PYGZsq{\textquotesingle}

\begin{document}

\maketitle
\tableofcontents
\phantomsection\label{index::doc}



\chapter{Как работать в Kanboard}
\label{index:using-kanboard}\label{index:documentation}

\section{Введение}
\label{index:introduction}\begin{itemize}
\item {} 
Что такое Kanban?

\item {} 
Kanban против Todo списков и Scrum

\item {} 
Где можно использовать Kanboard

\end{itemize}


\section{Использование доски}
\label{index:using-the-board}\begin{itemize}
\item {} 
Доска, Календарь, Список и Гант представления

\item {} 
Компактное или развернутое отображение задач

\item {} 
Горизонтальная прокрутка и компактный вид

\item {} 
Отображение и скрытие колонок

\end{itemize}


\section{Работа с проектами}
\label{index:working-with-projects}\begin{itemize}
\item {} 
Типы проектов

\item {} 
Создание проектов

\item {} 
Редактирование проектов

\item {} 
Публичные доски и задачи

\item {} 
Автоматизация процессов

\item {} 
Права доступа к проекту

\item {} 
Дорожки

\item {} 
Календарь

\item {} 
Аналитика

\item {} 
Диаграмма Ганта для задач

\item {} 
Диаграмма Ганта для проектов

\item {} 
Пользовательские фильтры

\end{itemize}


\section{Работа с задачами}
\label{index:working-with-tasks}\begin{itemize}
\item {} 
Создание задач

\item {} 
Закрытие задач

\item {} 
Дублирование и перенос задач

\item {} 
Добавление снимка экрана (скриншота)

\item {} 
Ссылки на задачу

\item {} 
Перемещения

\item {} 
Отслеживание времени

\item {} 
Повторяющиеся задачи

\item {} 
Создание задач через email

\item {} 
Подзадачи

\item {} 
Аналитика для задач

\item {} 
Ссылка на пользователя

\end{itemize}


\section{Работа с пользователями и группами}
\label{index:working-with-users-and-groups}\begin{itemize}
\item {} 
Роли

\item {} 
Типы пользователей

\item {} 
Управление группами

\item {} 
Управление пользователями

\item {} 
Уведомления

\item {} 
Двухуровневая аутентификация

\end{itemize}


\section{Настройки}
\label{index:settings}\begin{itemize}
\item {} 
Горячие клавиши

\item {} 
Настройки приложения

\item {} 
Настройки проекта

\item {} 
Настройка Доски

\item {} 
Настройки календаря

\item {} 
Настройка ссылок

\item {} 
Курсы валют

\end{itemize}


\section{Встроенные возможности}
\label{index:integrations}\begin{itemize}
\item {} 
iCalendar подписки

\item {} 
RSS/Atom подписки

\item {} 
Json-RPC API

\item {} 
Webhooks

\item {} 
Плагины

\end{itemize}


\section{Дополнительно}
\label{index:more}\begin{itemize}
\item {} 
Синтаксис расширенного поиска

\item {} 
Интерфейс командной строки

\item {} 
Руководство по синтаксису

\item {} 
Защита от Brute force

\item {} 
Часто задаваемые вопросы

\end{itemize}


\chapter{Технические детали}
\label{index:technical-details}

\section{Инсталяция}
\label{index:installation}\begin{itemize}
\item {} 
Требования

\item {} 
Инструкция по инсталяции

\item {} 
Обновление Kanboard до новой версии

\item {} 
Инсталяция на Ubuntu

\item {} 
Инсталяция на Debian

\item {} 
Инсталяция на Centos

\item {} 
Инсталяция на OpenSuse

\item {} 
Инсталяция на FreeBSD

\item {} 
Инсталяция на Windows Server и IIS

\item {} 
Инсталяция на Windows Server и Apache

\item {} 
Инсталяция на Heroku

\item {} 
Запуск Kanboard под Docker

\item {} 
Запуск Kanboard под Vagrant

\item {} 
Запуск Kanboard на Cloudron

\item {} 
Запуск Kanboard на Nitrous

\end{itemize}


\section{Настройка}
\label{index:configuration}\begin{itemize}
\item {} 
Ежедневные фоновые задачи

\item {} 
Конфигурационный файл

\item {} 
Переменные окружения

\item {} 
Настройка email

\item {} 
Переопределение URL

\item {} 
Директория плагинов

\end{itemize}


\section{База данных}
\label{index:database}\begin{itemize}
\item {} 
База данных Sqlite

\item {} 
Как использовать Mysql

\item {} 
Как использовать Postgresql

\end{itemize}


\section{Аутентификация}
\label{index:authentication}\begin{itemize}
\item {} 
LDAP аутентификация

\item {} 
Синхронизация групп LDAP

\item {} 
Изображения из профиля LDAP

\item {} 
Параметры LDAP

\item {} 
Пример конфигурации LDAP

\item {} 
Аутентификация Reverse proxy

\end{itemize}


\section{Участие в проекте}
\label{index:contributors}\begin{itemize}
\item {} 
Руководство для участников проекта

\item {} 
Переводы на другие языки

\item {} 
Стандарты при написании кода

\item {} 
Выполнение тестов

\item {} 
Создание assets

\end{itemize}

\href{https://github.com/fguillot/kanboard/tree/master/doc}{Документация} написана в формате \href{https://ru.wikipedia.org/wiki/Markdown}{Markdown}. Если вы желаете улучшить документацию - пошлите pull-request.
* \href{https://github.com/hairetdin/kanboard-doc-ru}{Проект перевода документации Канборд на русский язык}. \href{http://kanboard.ru/doc/}{Русская документация Канборд в формате html}.



\renewcommand{\indexname}{Алфавитный указатель}
\printindex
\end{document}
